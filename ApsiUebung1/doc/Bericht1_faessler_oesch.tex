\documentclass[12pt]{scrartcl}
\usepackage[utf8]{inputenc} 
\usepackage{fancyhdr, graphicx}
\usepackage[german]{babel}
 \usepackage[scaled=0.92]{helvet}
 \usepackage{enumitem}
 \usepackage{parskip}
 \usepackage{lastpage} % for getting last page number
 \renewcommand{\familydefault}{\sfdefault}
 
\usepackage{color}
\usepackage{xcolor}
\usepackage{listings}

\usepackage{caption}
\DeclareCaptionFont{white}{\color{white}}
\DeclareCaptionFormat{listing}{\colorbox{gray}{\parbox{\textwidth}{#1#2#3}}}
\captionsetup[lstlisting]{format=listing,labelfont=white,textfont=white}

\usepackage{listings}
  \usepackage{courier}
 \lstset{
         basicstyle=\footnotesize\ttfamily, % Standardschrift
         numbers=left,               % Ort der Zeilennummern
         numberstyle=\tiny,          % Stil der Zeilennummern
         %stepnumber=2,               % Abstand zwischen den Zeilennummern
         numbersep=5pt,              % Abstand der Nummern zum Text
         tabsize=2,                  % Groesse von Tabs
         extendedchars=true,         %
         breaklines=true,            % Zeilen werden Umgebrochen
         keywordstyle=\color{red},
    		frame=b,         
 %        keywordstyle=[1]\textbf,    % Stil der Keywords
 %        keywordstyle=[2]\textbf,    %
 %        keywordstyle=[3]\textbf,    %
 %        keywordstyle=[4]\textbf,   \sqrt{\sqrt{}} %
         stringstyle=\color{white}\ttfamily, % Farbe der String
         showspaces=false,           % Leerzeichen anzeigen ?
         showtabs=false,             % Tabs anzeigen ?
         xleftmargin=17pt,
         framexleftmargin=17pt,
         framexrightmargin=5pt,
         framexbottommargin=4pt,
         %backgroundcolor=\color{lightgray},
         showstringspaces=false      % Leerzeichen in Strings anzeigen ?        
 }
 \lstloadlanguages{% Check Dokumentation for further languages ...
         %[Visual]Basic
         %Pascal
         %C
         %C++
         %XML
         %HTML
         Java
 }
 
 \fancypagestyle{firststyle}{ %Style of the first page
 \fancyhf{}
 \fancyheadoffset[L]{0.6cm}
 \lhead{
 \includegraphics[scale=0.8]{./fhnw_ht_e_10mm.jpg}}
 \renewcommand{\headrulewidth}{0pt}
 \lfoot{Institute of computer science,\linebreak www.fhnw.ch }
}

\fancypagestyle{documentstyle}{ %Style of the rest of the document
 \fancyhf{}
 \fancyheadoffset[L]{0.6cm}
\lhead{
 \includegraphics[scale=0.8]{./fhnw_ht_e_10mm.jpg}}
 \renewcommand{\headrulewidth}{0pt}
 \lfoot{\thepage\ / \pageref{LastPage} }
}

\pagestyle{firststyle} %different look of first page
 
\title{Warum der Wert von $n$ in $2^n$ wichtig ist}

 \begin{document}
 \maketitle
 \thispagestyle{firststyle}
 \pagestyle{firststyle}
 \begin{abstract}
 \begin{center}
  Fabio Oesch \& Jan Fässler
 \end{center}
 \vspace{0.5cm}
\hrulefill
\end{abstract}

 \pagestyle{documentstyle}
 \tableofcontents
 \pagebreak
\section{Aufgabenstellung}
Konstruieren Sie zwei Briefe an Alice, einen (orginal) für Bob und einen (gefälscht) für Alice, die aber den gleichen Hashwert haben. Da die Fälschung etwas für Sie einbringen soll, ersetzen Sie im Brief an Alice die Kontonummer 222-1101.461.12 durch Ihre eigene: 202-1201.262.10. Sie haben freilich bei der gleichen Bank ein Konto mit dieser Nummer eröffnet.
\section{Software Aufbau}
Das Projekt wurde in 2 Klassen unterteilt. Die Klasse HashingMachine ist für das Hashing zuständig. Die Klasse LabOne ist für das einlesen der Daten und das Durchprobieren der Hashingmethoden zuständig.
\subsection{HashingMachine}
Diese Klasse hasht den Vorgegebenen Text mit der Hashfunktion die wir von der Aufgabe erhalten haben.
\subsubsection{preprocess}
Die Methode preprocess(byte[] input) strukturiert den Input wie er nachher benötigt wird.
\lstinputlisting[label=Structure the input,caption=Structure the input]{inputstructure.java}
Die erste Linie kopiert den input in den Ouput. Linien 2 und 3 sind für das Padding verantwortlich. Die Linie 4 kopiert die Länge in den Output.
\subsubsection{create}
\lstinputlisting[label=Hashing,caption=Hashing]{create.java}
Auf der Zeile .. wird der cipher mit dem Initialvektor initialisiert.\\
Xor Magic von altem und neuem Zeugs\\
Da dies ein Blockcipher ist wird nun ein neuer hash benutzt mit der der Text verschlüsselt wird.\\
\subsection{LabOne}
Main Klasse in der, die Hashfunktion geknackt wird.
\subsubsection{main}
In dieser Klasse können wir Einstellungen für unsere Hashfunktion vornehmen.
\subsubsection{resolveCombination}
Je nachdem wie der Integer ist von combination wird ein String generiert, mit dem Template, für die er die verschiedenen Optionen von options.ini einsetzt.
\subsubsection{createVariation}
Macht das selbe wie resolveCombination jedoch wird stattdessen der Hashwert als Integer zurückgegeben.
\subsubsection{generateHashes}
\lstinputlisting[label=Hashing Knacken,caption=Hashing Knacken]{hashtest.java}
Methode welche überprüft, ob man den gleichen Hashwert erhalten hat. Man überprüft, ob man die generierten fakeCombination's eine Kollision mit den dem erstellten orginalCombination hat.


 
 \end{document}